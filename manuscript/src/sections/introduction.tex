\section{Chapter 1 - Introduction}\label{sec:introduction}
Water has become one of the most important and contested natural resources of the 21st century. Increasing population growth, accelerating urbanization, and the intensifying effects of climate change are exerting significant pressure on freshwater systems worldwide (American Meteorological Society 2017)\cite{AMS2017}. Dams have historically played a critical role in water management, providing storage for irrigation, hydropower generation, domestic supply, and flood control. Globally, they are regarded as cornerstones of national development strategies, yet their design and expansion decisions require careful balancing of social, economic, and environmental considerations \cite{Schmitt2024,AMS2017}.    

In North Africa, and particularly in Morocco, the challenge of water scarcity is especially pronounced. Morocco is among the most water-stressed countries in the region, with per capita renewable water resources steadily declining over the last decades, from about $2,560m_3$ in 1960 to roughly $620m_3$ in 2020. THis reduction is attributed to reduced rainfall, rising demand, and sedimentation in existing reservoirs \cite{WorldBank2023}. Dams remain central to Morocco's national water policy, as they underpin agricultural productivity, food security, and energy diversification. The government has therefore invested heavily in dam infrastructure expansion, with over 150 large dams already constructed, and several additional projects planned \cite{TradeGov2024}. However, the benefits of such infrastructure are accompanied by substantial trade-offs related to land use, environmental sustainability, and local socio-economic impacts.

Selecting suitable sites for future dams thus constitutes a complex strategic decision problem. Beyond hydrological and engineering feasibility, the siting of dams must also account for population distribution, access to transport networks, farmland protection, and broader ecological constraints. Conventional single-objective approaches often fall short in capturing these competing dimensions. As a result, there has been a shift toward multi-criteria decision-making (MCDM) methods, which provide structured frameworks to integrate diverse quantitative and qualitative factors into dam site evaluations \cite{TradeGov2024,Minatour2015}. In this study, we position dam site selection  for expansion in Morocco within this broader discourse of sustainable and multi-objective water infrastructure planning.

\subsection{Problem of Dam Site Selection}
While dams are vital for Morocco's long-term water security and economic growth, the process of selecting their sites involves a web of conflicting objectives and constraints. On one hand, governments and planners seek to maximize storage capacity, agricultural productivity, and energy generation. On the other, environmental and social considerations, such as farmland preservation, displacement risks, ecological integrity, and equitable access, place strong constraints on dam siting decisions \cite{Wang2021,Zerdeb2025}. These competing goals transform site selection from a straightforward engineering task into a multi-objective decision problem that requires balancing diverse interests under uncertainty.

Traditional approaches to dam site evaluation have often prioritized hydrological suitability or cost efficiency, using single-objective models that emphasize technical feasibility. However, these methods tend to underrepresent the wider social and environmental impacts, leading to decisions that may be technically sound but socially or ecologically unsustainable. To overcome these limitations, researchers and practitioners have increasingly adopted multi-criteria decision-making (MCDM) frameworks, which explicitly incorporate heterogeneous factors into decision analysis \cite{Wang2021,Hagos2022}. MCDM methods enable decision makers to consider trade-offs among economic, environmental, and social criteria, while also accommodating input from multiple stakeholders.

In Morocco, these complexities are heightened by the diversity of candidate dam sites and the country's urgent need for expansion under resource constraints. From the 28 candidate sites considered in this study, three is to be selected, making the optimization problem both discrete and sensitive to value judgments about which criteria should dominate. This tension underscores the importance of adopting robust decision-support tools capable of producing transparent, justifiable, and adaptable recommendations. It is in this context that the present work turns to \gls{GP}, a family of MCDM techniques especially suited to structuring and solving problems where multiple, potentially conflicting objectives must be addressed simultaneously.

\subsection{Multi-Criteria Decision-Making}
Decision-making in real-world contexts rarely revolves around a single objective. Governments, businesses, and communities are frequently confronted with situations where they must balance conflicting goals, for example, maximizing economic returns while minimizing environmental damage, or ensuring technical efficiency while promoting social equity. To address such complexities, scholars and practitioners have developed Multi-Criteria Decision-Making (MCDM) as a structured scientific approach that enables systematic evaluation of alternatives when trade-offs are unavoidable \cite{Aruldoss2013,Taherdoost2023}. Far from being a purely theoretical construct, MCDM has become a practical decision-support tool that reflects the realities of modern governance and resource management \cite{HUANG2011}.

At its core, Multi-Criteria Decision-Making (MCDM) refers to a family of quantitative and qualitative techniques designed to support decisions that involve multiple, often conflicting, evaluation criteria \cite{Aruldoss2013,Taherdoost2023}. Unlike simple optimization models, which concentrate on maximizing or minimizing a single objective, MCDM provides a structured way for decision-makers to balance competing priorities \cite{jones2010}. In practice, this means that MCDM captures the reality that most real-world choices are about trade-offs rather than absolutes. In this sense, it can be seen as a form of collective reasoning: just as a group of individuals brings diverse perspectives to arrive at a shared judgment, MCDM integrates diverse evaluation criteria into a coherent and balanced decision outcome \cite{Borges2020,Cinalli2015}.

Over time, Multi-Criteria Decision-Making (MCDM) has developed into an interdisciplinary field, drawing insights from mathematics, economics, computer science, and the social sciences. The rapid growth of computational power has accelerated this evolution, making it possible to design more sophisticated approaches. In particular, fuzzy MCDM methods have been introduced to address uncertainty in decision environments \cite{LIANG1999,Mardani2015}, while hybrid approaches now integrate MCDM with artificial intelligence and machine learning to enhance accuracy and adaptability \cite{Arabameri2020,PHAM2021}. As a result, MCDM has moved beyond being a theoretical tool to become a cornerstone of modern decision science, widely applied to some of today's most complex real-world challenges.

MCDM techniques have demonstrated remarkable versatility across diverse fields. In engineering and environmental planning, they provide structured frameworks for prioritizing design alternatives and evaluating trade-offs in infrastructure development, environmental impact assessments, and land-use planning \cite{karakus2022,Dirie2024}. MCDM has been employed in energy systems to compare renewable energy technologies, identify suitable sites for facilities, and support the transition toward low-carbon strategies. Business and finance also has use cases in assisting managers in evaluating investment portfolios, assessing operational risks, and formulating strategic policies \cite{TAMIZ1998}. Beyond these traditional domains, MCDM has also become increasingly prominent in sustainability research, where decision-makers must balance economic growth, ecological preservation, and social welfare in an integrated manner \cite{ettazarini2021,Mardani2015}. Collectively, these applications illustrate how MCDM adapts to the specific needs of each context while maintaining its role as a systematic tool for rational decision-making.

These applications highlight the very nature of the challenges MCDM is designed to address: decisions with multiple stakeholders, competing objectives, incomplete information, and long-term uncertainties. In this sense, MCDM resonates with the philosophy of collective intelligence, where diverse contributions must be synthesized into a coherent solution\cite{Cinalli2015}. Just as collective intelligence seeks to prevent dominance by a single actor in group decision-making, MCDM provides a safeguard against the dominance of a single criterion in technical evaluations. This parallel underscores MCDM's role not only as a computational tool but also as a conceptual bridge between quantitative rigor and inclusive decision-making.   

The importance of MCDM has grown in recent decades due to the increasing complexity of global challenges. Climate change, resource scarcity, and sustainable development goals all require decisions that balance competing priorities. For instance, governments must decide how to allocate limited water supplies across agriculture, energy, and domestic use; companies must balance profitability against environmental and social responsibility; and communities must weigh development needs against cultural and ecological preservation \cite{KUMAR2017596,PortnerIPCC2022}. Traditional single-objective decision tools fall short in these contexts, while MCDM offers a structured and transparent process for evaluating trade-offs.

MCDM provides a rigorous yet flexible decision-support framework, particularly well-suited to wicked problems—those with no single optimal solution but multiple competing pathways. This makes it a powerful tool for addressing Morocco's dam site investment challenge, where economic, social, and environmental goals must all be considered simultaneously under conditions of uncertainty.

\subsection{MCDM in Water Resource Management and Dam Site Selection}
The complexity of water resource management makes it a prominent field for the application of Multi-Criteria Decision-Making (MCDM). Early studies in this domain emphasized technical and economic feasibility, particularly in irrigation planning, watershed management, and hydropower development \cite{POHEKAR2004,Romanelli2018}. However, these approaches often overlooked environmental and social dimensions, limiting their comprehensiveness.

From the mid-2000s onward, the literature reflects a growing integration of Geographic Information Systems (GIS) with MCDM to enable spatially explicit decision frameworks. For example, \cite{Romanelli2018} combined GIS and AHP to identify hydropower locations in Brazil, while \cite{karakus2022} applied GIS-based multi-criteria analysis for dam site suitability in Turkey. These studies highlighted how spatial integration of hydrological, geological, and socio-economic data strengthens the robustness of site evaluations.

Uncertainty in hydrological and climate conditions has also driven the development of fuzzy and probabilistic MCDM methods. \cite{LIANG1999} pioneered fuzzy extensions of MCDM based on ideal and anti-ideal concepts, and subsequent reviews \cite{Mardani2015} show that fuzzy MCDM techniques have become widely used for handling ambiguity in water resource allocation and dam planning. Recent studies extend this trend by employing hybrid models that combine MCDM with optimization algorithms or artificial intelligence. \cite{PHAM2021}, for instance, integrated hybrid artificial intelligence models with multi-criteria decision analysis to improve flood risk assessment in Vietnam.

The evolution of criteria in dam site selection is another clear trend. While early models emphasized economic and engineering parameters, contemporary studies increasingly incorporate environmental and social concerns such as biodiversity protection, resettlement impacts, and ecosystem trade-offs \cite{Dirie2024,karakus2022}. Reviews of the literature highlights four broad developments. The widespread adoption of GIS for spatial analysis, a shift toward sustainability criteria alongside engineering and economic factors, expanded use of fuzzy and probabilistic approaches to address uncertainty, and the integration of MCDM with machine learning and optimization for greater accuracy.

Despite these advances, several gaps persist. The selection and weighting of criteria remain highly subjective and context-dependent, creating challenges for replicability \cite{Belton2002,Mardani2015}. Data scarcity, especially in developing regions, further constrains model precision and reliability \cite{POHEKAR2004,Dirie2024}. Moreover, most studies emphasize projects such as new dams or small hydropower, while fewer address the optimization of existing infrastructure, which is equally critical under climate and fiscal constraints \cite{KUMAR2017596,Romanelli2018}. Finally, there is limited application of these frameworks in North African contexts, despite acute water scarcity and reliance on dams in countries such as Morocco \cite{ettazarini2021}.

This gap is particularly significant for Morocco, where the central challenge is not simply the identification of new dam sites but the prioritization of existing infrastructure for expansion and rehabilitation under competing economic, environmental, and social pressures. Addressing this gap requires the adoption of systematic, context-specific MCDM frameworks that explicitly integrate sustainability goals with fiscal and climate realities.

\subsection{Collective Intelligence in MCDM}
Multi-Criteria Decision-Making MCDM offers a family of methodologies—ranging from value measurement methods such as the Analytic Hierarchy Process (AHP) and Analytic Network Process (ANP), to outranking methods such as \gls{ELECTRE} and \gls{PROMETHEE}, to distance-based techniques like \gls{TOPSIS}, and to mathematical programming approaches such as \gls{GP} and Multi-Objective Linear Programming (MOLP)\cite{Belton2002,Aruldoss2013}. Each methodology provides a different mechanism for balancing conflicting objectives: value measurement methods rely on hierarchical structuring and subjective weights, outranking methods emphasize pairwise comparison and preference thresholds, distance-based approaches compare alternatives to ideal/anti-ideal solutions, and mathematical programming models optimize across multiple objectives under explicit constraints.

Theories of CI, as discussed by \cite{Woolley2010}, highlight that groups outperform individuals when three conditions are met: diversity of perspectives, independence of judgments, and effective aggregation mechanisms. When viewed through this lens, MCDM methodologies can be understood as formal aggregation mechanisms that operationalize these principles. AHP and ANP capture diversity through structured weighting, outranking methods allow pluralism of thresholds and vetoes, and fuzzy/gray extensions address uncertainty in human judgments \cite{Mardani2015,LIANG1999}. However, most of these methods are either too dependent on subjective weights or lack iterative feedback loops that resemble the adaptive, deliberative, and iterative nature of CI systems \cite{Cinalli2015}.

 \gls{GP} stands out as the MCDM methodology most aligned with CI theories. Like collective intelligence, \gls{GP} does not aim for a single optimal solution but rather for a 'satisficing' compromise that balances multiple, often conflicting, goals. Just as groups rarely arrive at unanimous “optimal” outcomes but instead reach negotiated compromises through deliberation, \gls{GP} models minimize deviations from a set of priority-ranked or weighted goals rather than forcing one criterion to dominate \cite{jones2010}. Moreover, \gls{GP} frameworks are inherently flexible: they allow the integration of diverse stakeholder goals, adjustment of priorities, and iterative recalibration—mirroring the feedback-driven and inclusive character of collective intelligence processes \cite{Borges2020}.

While many MCDM methods resonate with elements of CI,  is the methodology that most closely embodies its spirit, particularly in contexts where diverse goals must be reconciled rather than hierarchically imposed. For this reason, our study adopts four distinct  models—Weighted  \gls{WGP}, Compromise  \gls{CGP}, Extended  \gls{EGP}, and \gls{LGP}, to investigate how different compromise structures capture the dynamics of collective decision-making in complex, multi-criteria contexts.

\subsection{Goal Programming}
\gls{GP} is particularly well suited for addressing complex infrastructure planning problems such as dam site selection, where multiple and often conflicting objectives must be balanced. Unlike single-objective optimization, which collapses diverse concerns into a single aggregate function, \gls{GP} preserves the multidimensional nature of decision-making \cite{CHANG2007}. It does so by minimizing deviations from multiple goals, thereby offering solutions that reflect a compromise among technical, economic, environmental, and social considerations. In recent years, \gls{GP} models have been applied in contexts with strong stakeholder conflicts and ecological constraints, creating more sustainable decision pathways \cite{Castro2021}.

A further advantage of \gls{GP} lies in its flexibility. The method allows for different formulations that align with distinct decision‑making philosophies. The \gls{WGP} model facilitates explicit trade-offs between goals through assigned weights, making it well suited for contexts where objectives can be expressed in relative importance \cite{JONES2011}. \gls{LGP}, by contrast, imposes a strict hierarchy of priorities, ensuring that higher-order goals are fully satisfied before lower-order ones are considered \cite{jones2010}. \gls{CGP} emphasizes fairness by minimizing the maximum deviation across all goals, producing outcomes that are more equitable among competing objectives \cite{JONES2011}. Finally, the \gls{EGP} model generalizes the \gls{GP} framework by introducing additional parameters that penalize deviations asymmetrically, thereby capturing flexible stakeholder preferences and allowing more nuanced solutions \cite{jones2010}

Taken together, these four formulations enable decision makers to examine the dam site selection problem through multiple lenses: balanced compromise \gls{WGP}, priority satisfaction \gls{LGP}, fairness \gls{CGP}, and flexibility \gls{EGP}. In this study, each formulation is mapped to a distinct sub-question concerning how Morocco's dam expansion strategy might weigh competing objectives. By doing so, the analysis not only identifies feasible dam site combinations, but also generates a spectrum of alternatives reflecting different governance and policy orientations. This multimodel approach also strengthens the robustness of final recommendations, as it allows comparisons across frameworks and provides insights into the conditions under which different dam sites emerge as optimal.

