\section{Chapter 5 - Conclusions}\label{sec:conclusions}
This study applied Goal Programming \gls{GP} to the problem of selecting three dam sites from a set of twenty-eight candidates under multiple, and sometimes conflicting, objectives. \gls{GP} was chosen because of its ability to balance competing economic, social, environmental, and technical concerns in a structured and transparent way. Unlike single-objective optimization, \gls{GP} offers multiple formulations that allow decision makers to incorporate their preferences through weighting, prioritization, and fairness-driven rules. Moreover, \gls{GP} aligns closely with principles of collective intelligence, providing a decision framework that accommodates diverse viewpoints and contextual constraints while still generating actionable recommendations.  

To fully explore the dam site selection problem, four \gls{GP} variants were employed, each addressing a distinct sub-question. The \gls{WGP} model asked: what is the most balanced solution when all objectives are weighted simultaneously? The Lexicographic \gls{LGP} model investigated: what solution emerges when objectives are ranked by strict priority? The CGP model examined: what solution minimizes the maximum deviation, thereby ensuring fairness across objectives? Finally, the Extended Goal \gls{EGP} model explored: how do solutions change when large deviations are penalized more heavily, revealing asymmetric trade-offs? Together, these formulations ensured that the problem was analyzed not from a single viewpoint, but across different logics of compromise, priority, fairness, and flexibility.

\gls{MCDM} was applied specifically to five selected targets: population, residence distance, farmland distance, nearest road, and farmland area. These criteria were identified as the most critical sociotechnical and environmental considerations from earlier screening of possible objectives. Focusing on these five allowed for a tractable yet comprehensive representation of the competing goals most relevant to dam placement. The purpose of using MCDM here was not only to reveal the trade-offs between these key targets, but also to compare how different \gls{GP} models interpret and resolve these trade-offs, thereby supporting more transparent decision-making.

Sensitivity analysis was then conducted in two parts to test both the robustness and the reliability of the results. Weight sensitivity analysis examined how solutions shifted when the relative importance of objectives was perturbed, revealing which dam sites were consistently robust (selected across many scenarios) and which were sensitive to weight changes. Target sensitivity analysis, in contrast, evaluated the models under changing target levels, demonstrating that while WGP and LGP were stable, CGP and EGP displayed greater variability. These experiments were designed not only to test model stability, but also to guide decision makers by highlighting which recommendations remain credible even under uncertainty.

The combined findings suggest that dams 18, 19,  and 20 form a consistently robust core set, appearing frequently across models and scenarios. However, a more nuanced recommendation emerges from the sensitivity results: while these three sites are highly reliable, site 28 and, to a lesser extent, sites 21 and 23, emerge as meaningful alternatives when particular objectives—especially environmental or social priorities—are given greater weight. Thus, the final recommendation is not a single rigid solution, but a set of robust core dams (18, 19, 20) complemented by flexible alternatives (28, 21, 23) that can be emphasized depending on contextual preferences. This flexibility, supported by \gls{GP} and sensitivity analysis, strengthens the decision-making process by balancing stability with adaptability to local policy and stakeholder priorities.

Beyond the specific case of dam site selection, this study demonstrates the broader value of \gls{GP} in infrastructure planning under competing objectives. By combining multiple \gls{GP} formulations with systematic sensitivity analysis, the research illustrates how quantitative models can be translated into robust yet flexible recommendations that accommodate diverse stakeholder priorities. This integration of MCDM with sensitivity testing thus strengthens the link between optimization models and practical, policy-relevant decision support.