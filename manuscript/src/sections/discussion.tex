\section{Discussion}\label{sec:discussion}
Two clear patterns emerge from the experiments a \textit{stable core} of sites that persist across modeling choices and a \textit{flexible margin} whose inclusion depends on policy emphasis.

\paragraph{Stable core: dams 18, 19, and 20.}
These three recur because, taken together, they satisfy complementary benefit goals (maximize) and cost goals (minimize) under the budget and cardinality constraints:
\begin{itemize}
\item \textbf{Dam 18} combines very high rainfall (30.36; benefit~$\uparrow$) with a very small reservoir area (0.02; cost~$\downarrow$). Although its distance to settlement (0.35; benefit~$\uparrow$) and road distance (3.49; cost~$\downarrow$) are not standout, the climate supply plus low inundation footprint make it difficult to dislodge when goals are balanced. The budget cost is modest (157.50).
\item \textbf{Dam 19} is strong on low temperature (16.26; cost~$\downarrow$), very small reservoir area (0.02; cost~$\downarrow$), good rainfall (22.99; benefit~$\uparrow$), and short farm and road distances (0.90 and 0.93; cost~$\downarrow$). Although its distance to settlement (0.81; benefit~$\uparrow$) and farmland area (0.62; benefit~$\uparrow$) are moderate, the overall multi-criterion balance---especially on the cost side---drives repeated selection. The budget cost is modest (158.70).
\item \textbf{Dam 20} combines high rainfall (33.23; benefit~$\uparrow$), low temperature (16.92; cost~$\downarrow$), good farmland area (15.55; benefit~$\uparrow$), and large capacity (62; benefit~$\uparrow$). Distances are mid-range (residence 3.63; farm 2.47; road 1.80), but its supply and land-benefit profile keep it in the set when priorities are hierarchical or balanced. The budget cost is moderate (166.50).
\end{itemize}
Overall, dams 18--20 form a \textit{least-regret set}: even when weights or targets move, there is no single criterion that decisively breaks any of them, and jointly they span climate/resource adequacy (rainfall, capacity), low inundation footprint (small reservoir area for 18--19), and operational feasibility (reasonable access costs), all within budget.

\paragraph{Flexible margin: dams 28, 21, and 23.}
These sites function as \textit{policy levers} that enter when emphasis tilts toward particular social--environmental objectives:
\begin{itemize}
\item \textbf{Dam 28} offers extremely low temperature (15.56; cost~$\downarrow$), tiny reservoir area (0.01; cost~$\downarrow$), high distance to settlement (11.30; benefit~$\uparrow$), and short farm distance (0.93; cost~$\downarrow$). It is weaker on rainfall (7.54) and road distance (3.30; cost~$\downarrow$). It appears when settlement buffers and inundation limits are prioritized over hydrologic supply or when road access is less penalized.
\item \textbf{Dam 21} combines a very small reservoir area (0.01; cost~$\downarrow$), good rainfall (25.07; benefit~$\uparrow$), very high distance to settlement (13.91; benefit~$\uparrow$), and strong farmland area (27.25; benefit~$\uparrow$). Its road distance is large (5.15; cost~$\downarrow$) and farm distance moderate (2.97), so it becomes attractive when social buffers and farmland benefits dominate access costs.
\item \textbf{Dam 23} has outstanding farmland area (91.07; benefit~$\uparrow$) and very short farm distance (0.24; cost~$\downarrow$), making it a natural choice when agricultural benefit is emphasized. Its rainfall (2.53) and distance to settlement (0.24) are low, so it recedes when supply or settlement buffers tighten.
\end{itemize}

\paragraph{Objective values versus robustness}
In the WGP sensitivity experiments, some runs produced numerically smaller objective values than the base (0.26--2.50 versus 14.93).\ref{fig:wgp_sa_lollipop_fval} Since the original WGP was performed with constant weights (1), the observation of smaller objective values, with small deviations in the sensitvity analyses shows that a properly defined weight set could possibly fall reduce the original objective value. On the other hand, because reweighting changes the unit scale of the weighted-sum objective, these values may not directly comparable for overall goodness across runs and may indicate a shift in emphasis. A more interpretable indicator is selection robustness, for example $R = \frac{runs selected}{runs}$: dams 18--20 exceed a reasonable robustness threshold ($R >= 0.6$) , hence the \textit{core} dams 28, 21, and 23 exhibit context-sensitive inclusion, with $R$ rising in scenarios that favor settlement buffers and inundation limits (28, 21) or agricultural benefit (23). In LGP, as expected under lexicographic logic, top-tier priorities stabilize the choice set, while variation appears in lower-tier residuals. \ref{tab:lgpPriorityObjectives}

\paragraph{Practical implication.}
The models do not prescribe a single rigid optimum. They articulate a \textit{decision space} with a stable backbone (18--20) and policy-tunable complements (28, 21, 23). This is advantageous for planning under uncertainty. Decision-makers can commit to the core for baseline resilience and activate the margin to reflect evolving social (settlement buffers), environmental (inundation footprint, temperature), or agricultural (farmland access/area) priorities, without violating budget or feasibility.


% --------------------------------------------
\subsection{Comparison with existing literature}
This study's findings align with several trajectories in the multi-criteria decision-making (MCDM) literature. First, prior surveys document the dominance of value-measurement and outranking approaches (e.g., AHP/ANP, TOPSIS, ELECTRE, PROMETHEE) in infrastructure and siting problems, often implemented with GIS to achieve spatially explicit screening and ranking. These methods have proven effective for structuring criteria hierarchies, eliciting weights, and visualizing trade-offs, and they remain the workhorse in environmental and regional planning applications \cite{ajis2013115,encyclopedia3010006}. In contrast, our work deploys \textit{four} goal programming (GP) formulations on the same decision set, thereby reframing site selection from a single deterministic ranking problem into a family of compromise-seeking models that embody distinct decision logics (balance, priority, fairness, blended) \cite{JonesTamiz2010_PracticalGP}.

Classic distance-to-ideal and outranking approaches are sensitive to weight and threshold settings. Fuzzy and grey extensions were introduced precisely to buffer this sensitivity by representing imprecision in preferences and data \cite{LIANG1999682,MARDANI20154126}. Our results similarly show that model choice and target specification matter, CGP and EGP, which emphasize fairness and blended objective structures, are more reactive to target shifts than WGP and LGP. Rather than adopting fuzzy sets, we operationalize uncertainty via multi-choice goals and explicit sensitivity analyses on both weights and targets. This strategy is consistent with calls in the survey literature to move beyond one-shot rankings toward robustness analysis and scenario exploration \cite{ajis2013115,encyclopedia3010006}.

The GP literature has long argued that satisficing with explicit deviation penalties is well suited to public decisions with competing objectives and hard constraints \cite{JonesTamiz2010_PracticalGP}. Our findings reinforce this view in the dam expansion context. WGP and LGP deliver \textit{stable} recommendations (a persistent core of sites) that are straightforward to communicate, while CGP/EGP surface tension points criteria that become bottlenecks when targets or scales change. This mirrors the theoretical distinction drawn in GP between weighted compromise, lexicographic priority satisfaction, and Chebyshev fairness, and it evidences the practical value of inspecting multiple GP formulations side-by-side rather than privileging a single model.

On criteria design, recent reviews emphasize the shift from purely technical-economic indicators toward sustainability and socio-spatial dimensions in siting (e.g., population exposure/access, land use, proximity/fragmentation), alongside persistent issues of criteria subjectivity and context dependence \cite{KUMAR2017596,Aruldoss2013}. Our criteria set follows this evolution: beyond hydrologic and structural capacity, we incorporate population (normalized), farmland access/area, and proximity to settlements and roads. The observed separation in our results between a stable backbone of sites and a flexible margin is consistent with the literature's observation that a few alternatives often dominate on multiple criteria, while a second tier becomes competitive when social or environmental weights increase \cite{Aruldoss2013}.

With respect to method-reporting practice, several surveys note that many MCDM applications stop at a single configuration without probing robustness, and that transparency about scaling, normalization, and parameter choices is frequently underdeveloped \cite{Aruldoss2013}. We address these concerns by (i) disclosing target construction and normalization; (ii) testing both weight and target perturbations; and (iii) summarizing robustness with selection frequencies (and, where relevant, overlap with the base solution). In this sense, the present study contributes an applied template for multi-model GP analysis with explicit robustness narration. This complements the broader MCDM trajectory toward uncertainty-aware, defensible decision support \cite{jones2010,Mardani2015}.

\subsection{Methodological Contributions}
This study makes three methodological contributions to the field of multi-criteria decision making (MCDM) and infrastructure planning. 

First, by employing four distinct Goal Programming (GP) formulations—Weighted, Lexicographic, Chebyshev, and Extended—the analysis demonstrates how alternative logics of compromise, priority, fairness, and flexibility can be operationalized within a single decision problem. This multi-model perspective moves beyond the dominance of single-method studies in dam planning and highlights the value of comparative modeling \cite{Belton2002,Aruldoss2013}.  

The integration of Multi-Choice Goal Programming (MCGP) into the WGP, CGP, and EGP formulations introduces a novel way of handling uncertainty in socio-technical targets such as population, farmland area, and road access. While earlier dam site studies have applied AHP, TOPSIS, or GIS-based analyses, few have explicitly incorporated flexible aspiration levels. This extension enhances realism by acknowledging that planning targets are rarely fixed and often evolve with policy or stakeholder negotiations.  

Third, the systematic use of sensitivity analysis, both weight and target-based, provides a robustness check that is often underdeveloped in dam site selection literature. By combining Dirichlet-sampled weight perturbations with scenario-based target variations, the study identifies robust core sites while mapping the conditions under which alternative sites gain prominence. This strengthens the credibility of recommendations by ensuring that they hold under a range of plausible assumptions.  

These methodological advances contribute to the broader MCDM literature by illustrating how GP can be adapted for complex, high-stakes water resource decisions in data-scarce and uncertainty-prone contexts such as Morocco.

\subsection{Theoretical Implications}
The findings also carry theoretical implications for the broader field of MCDM.  
By applying four Goal Programming (GP) variants to the same decision problem, the study illustrates how distinct logics of decision-making—balance (WGP), priority (LGP), fairness (CGP), and flexibility (EGP)—can coexist within a unified analytical framework \cite{JonesTamiz2010_PracticalGP}. This reinforces the view that no single MCDM model captures the full complexity of real-world trade-offs \cite{KUMAR2017596,Aruldoss2013}.  

Furthermore, the integration of multi-choice goals and sensitivity testing highlights the importance of treating decision-making as a process under uncertainty rather than a one-off optimization. In this sense, GP models resonate with collective reasoning theories by emphasizing compromise and adaptability over rigid optimality \cite{Borges2020}. Together, these insights position GP not only as a computational tool but also as a conceptual bridge between optimization models and inclusive, deliberative decision processes.

\subsection{Limitations and Future Work}
Despite its contributions, this study has some limitations.  
First, the analysis relies on available hydrological, climatic, and socio-economic datasets that, while comprehensive, may not fully capture ecological constraints such as biodiversity impacts or sedimentation dynamics. Data normalization and target setting, though carefully designed, remain partly subjective and context-dependent, reflecting a common challenge in MCDM applications \cite{Belton2002,Mardani2015}.  

Second, Goal Programming (GP) formulations are linear by design. While this makes them transparent and computationally tractable, real-world dam planning involves non-linear hydrological and ecological processes that may require more advanced hybrid or simulation-based approaches.  

Third, stakeholder preferences were represented indirectly through weights, priorities, and multi-choice targets rather than through participatory elicitation. As such, the models approximate but do not fully capture the deliberative dimension of decision-making.  

Future research should address these limitations by incorporating fuzzy or hybrid GP methods to account for non-linearity and uncertainty, integrating richer ecological and social datasets, and testing participatory frameworks where stakeholders co-define weights and aspiration levels. Extending the analysis to multi-period planning or rehabilitation of existing dams would also enhance policy relevance under climate and budgetary constraints.

\subsection{Closing Synthesis}
In summary, this study demonstrates that Goal Programming (GP) provides a rigorous yet flexible framework for addressing the multi-dimensional challenge of dam site selection in Morocco. By applying four GP variants alongside multi-choice extensions and systematic sensitivity analysis, the research reveals a clear structure of decision outcomes, a stable core of dams (18, 19, 20) that remain robust across models and scenarios, and a flexible margin of alternatives (28, 21, 23) that gain relevance under specific policy emphases.  

This dual structure—stability, when combined with adaptability, highlights the practical value of GP in contexts where policy must balance economic, social, environmental, and technical objectives under uncertainty. The analysis reinforces the view that infrastructure planning should not seek a single rigid optimum, but rather a decision space that accommodates evolving priorities and diverse stakeholder perspectives. Thus, the study advances both the methodological practice of MCDM and its theoretical positioning as a tool for inclusive and robust decision support in complex, resource-constrained environments.




